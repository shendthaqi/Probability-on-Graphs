\section{Basics of graph theory}

\begin{defn}[Graph]
    A graph \(G=(V,E)\) is a consists of 
    \begin{enumerate}
      \item \(V \neq \emptyset\) and
      \item \(E \ssq \set*{ \{u,v\}; u,v \in V}\)
    \end{enumerate}
    We call elements of $V$ vertices or nodes and elements of \(E\) edges or bonds. We write $v\sim u$ if \(\{u,v\}\in E\). A graph $G=(V,E)$ is called network if for $e=\{u,v\} \in V$, we assign a weight $c(u,v)>0$. For $v \in V$, we set $c(v)=\sum_{u \sim v}c(u,v)$. We further define 
    \begin{itemize}
      \item A vertex $v$ is called incident to an edge $e$ if $v \in e$
      \item The degree of a vertex $v$ is the number of edges incident to it: \(\operatorname{deg}v= \abs{\set*{e\in E; v\in e}} \).
      \item The graph $G$ is called locally finite if for all $v \in V$, \(\operatorname{deg}v<\infty\).
      \item For $u,v \in V$, $\pi =(x_1,\dots,x_N,)$ is called a path from $u$ to $v$ if $v=x_1,\, x_N=v$ and $ \{x_{n-1},x_n\} \in E$.  
      \item A path from $u$ to $u$ is called a cycle.
      \item A graph $G$ is called connected if for any $u,v \in V$ there is a finite path from $u$ to $v$.
    \end{itemize}
    
\end{defn}

ede


\begin{defn}[Product of graphs]
    Let $G_1, G_2$ be graphs. We define  
    \[G_1 \boxtimes G_2:= (V_1\times V_2, \cub{((x_1,y_1),(x_2,y_2))\fil \trm{either }(x_1=x_2 ,\,(y_1,y_2)\in E_2) \trm{ or } (y_1=y_2,\,(x_1,x_2)\in E_1)})\]
    as the product of $\GGG_1$ and \(\GGG_2\).
\end{defn}


\begin{defn}[Vertex simple and edge simple paths]
  Let $G$ be a graph. A path is a (possibly infinite) sequence of vertices $v_1,v_2,\dots$ such that $(v_n,v_{n+1})\in E$. 
  \begin{enumerate}
    \item If a path does not visit any vertex more that once, it is called a vertex simple path.
    \item If a path does not visit any edge more than once, it is called an edge simple path.
  \end{enumerate}
\end{defn}


\begin{defn}[Network]
  A network is a graph $G$ combined with a function $c:E\to [0,\infty)$ with if $c(u,v)=0$ if $(v,u)\notin E$. We call a network connected if $(V, \set*{e\in E; c(e)>0})$ is connected. A network is locally finite if for any \(v\in V\), we have \(c(v) = \sum_{u\sim v}c(u,v)<\infty\).  
\end{defn}


\begin{lem}[Example]
  $V=\ZZ$, $E= \cub{(v,u) \fil u\neq v\in V}$ with $c(v,u)=\frac{1}{\abs{u-v}^2}$.
\end{lem}


\begin{defn}[Cycle]
  A cycle is a path starting and ending in the same point. A cycle is simple if the only vertex appearing more than once is the first (and last) vertex.
\end{defn}


\begin{defn}[Induced graphs and networks]
  Let $G=(V,E)$ be a graph and let \(K\ssq V\). The induced graph is the graph \(G_K=(K, \cub{e\in E \fil e \ssq K})\). The induced network is defined similarly. 
\end{defn}

\begin{defn}[Connected component]
    Let $G$ be a graph a graph or a network. A connected component is a set $K$ of vertices such that $G_K$ is connected, but for every $K\subsetneq H $, $G_H$ is not connected.
\end{defn}

\begin{defn}[Contraction of graphs]
  Let $G=(V,E,c)$ be a graph (or network) and let $K\ssq G$. The connected graph (or network) $G/K$ is the graph $(V\setminus \cup \cub{z}, \cub{e \in E \fil e \ssq V\setminus K} \cup \cub{(v,z) \fil v \notin K, \exists u \in K: (v,u) \in E})$. For all $u,v \notin K$, take conductance \(c'(v,u) = c(v,u)\) for $v\notin K$ take $c(v,u)=\sum_{u\in K} c(v,u)$.
\end{defn}


\begin{defn}[Forest and tree]
    A forest is a graph with no (simple) cycles. A tree is a connected forest.
\end{defn}


\begin{remark}
    A contraction of an edge lets it "disappear" in a sense that the vertices are merged.
\end{remark}


\begin{defn}[Homomorphisms]
    Let $G_1,G_2$ be graphs. A function \(\phi: V_1 \to V_2\) is called a homomorphism if 
    \begin{enumerate}
      \item for all $(u,v) \in E_1$, we have that \((\phi(u), \phi(v)) \in E_2\)
    \end{enumerate}
    $\phi$ is called isomorphism if \(\phi\) is bijective, and both $\phi,\phi^{-1}$ are homomorphisms. $G_1,G_2$ are called isomorphic if such an isomorphism exists. An isomorphism $\phi$ is called an automorphism if $G_1=G_2$.
\end{defn}

\begin{defn}[Transitivity]
    A graph $G=(V,E)$ is called (vertex) transitive if for all \(v,u\in V\) there is an automorhpism \(\phi\) such that \(\phi(v)=u\). $G$ is called edge transitive if for every $e_1,e_2 \in E$ there is an automorphism $\phi(e_1)=e_2$.
\end{defn}

\begin{remark}
    There are graphs that are vertex transitive but not edge transitive.
\end{remark}

\begin{remark}
  Good source of example for transitive graphs: Let $G$ be a graph, and let $Y \ssq G\setminus\cub{\trm{unit}}$. Take the graph $V=G$ and $(u,v) \in E$ if there is $y\in Y$ such that $u=yv$ or $v=yu$.
\end{remark}


\subsection{Random walks on networks}

\begin{defn}[Simple rnadom walk on networks]
Let $G=(V,E,c)$ be a network and assume that $G$ is locally finite. The simple random walk on $G$ is the (discrete time) Markov chain with transition matrix
\[\Pi(x,y) = \frac{c(x,y)}{\sum_z c(x,z)} \cdot \ONE\left[\sum_z c(x,z) >0\right] + 1 \cdot \ONE\left[\sum_z c(x,z)=0\right] \]
A random walk on a locally finite graph $G=(V,E)$ is a random walk on the network $(V,E,1)$.
\end{defn}


\begin{remark}
    Question: Does \(\ZZ^d\) have a transient subgraph?\\
    Answer: No, every subgraph of a recurrent graph is recurrent. \\ 
    The main tool to provide the proof for the answer is based on the connection between the random walk and electrical networks.
\end{remark}


\begin{lem}
  Let $G=(V,E,c)$ be a finite connected network. For every $x\in V$, let \(p^x\) be the distribution of the random walk on $G$ started on $x$. Let $a,z \in V$. Let $\tau_a$ ($\tau_z$) be the first time hitting $a$ ($z$), and let $\tau_{a,z}$ be first time hitting either of them. Then, $\tau_a,\tau_z,\tau_{a,z}< \infty$.
\end{lem}

\begin{fact}
  Let $G=(V,E,c)$ be a finite connected network. We consider the simple random walk defined as above. Then, every state $x$ is recurrent. We have $\PP_x(\tau_a < \infty)= \mathbb{P}_x(\tau_z<\infty)=1$.
\end{fact}
\begin{proof}
For every $x$, let $\gamma_x$ be a path from $x$ to $a$. Let $h_x$ be the probability that the random walk takes the path $\gamma_x$ immediately starting in $x$. Let $h= \min_{x\in V}h_x>0$. Let $\mathcal{F}_n ) = \sigma(X_1, \dots, X_N), \, \mathcal{F}=\sigma(X_1, X_2, \dots)$. Then, $\set*{\tau_a < \infty } \in \FFF$. By Kolmogorov's 0-1-law, $\mathbb{P}_x \setround*{\tau_a<\infty ; \FFF} \in \set*{0,1 }$. We have \(\mathbb{P}_x \setround*{\tau_a < \infty  ; \mathcal{F}} = \lim_{n \to \infty} \mathbb{P}_x \setround*{\tau_a < \infty ; \mathcal{F}_n}\). We have \(\mathbb{P}_x \setround*{\tau_a < \infty ;\mathcal{F}_n} \geq h \). Thus, \(\mathbb{P}_x \setround*{\tau_a <\infty  ;\mathcal{F}}\geq h\). It follows from the 0-1-law, $\mathbb{P}_x \setround*{ \tau_a < \infty ;\mathcal{F}}=1$ and thus \(\mathbb{P}_x(\tau_a < \infty) = \mathbb{E} \left[\mathbb{P}_x \setround*{\tau_a <\infty; \mathcal{F}}\right]\). 
\end{proof}



\begin{defn}[Voltage function]
  The voltage function is defined as $v: V \to \RR, x \mapsto v(x):=p^x(\tau_{a,z}=\tau_a)$.
\end{defn}


\begin{lem}[Properties of the voltage function]
  Let $v: V \myxrightarrow[]{} \mathbb{R}, x \mapsto p^x(\tau_{a,z}=\tau_a)$ be the voltage function. Then, the following statements hold:
  \begin{enumerate}
      \item \(v(a)=1\)
      \item \(v(z)=0\)
      \item For $x\neq a,z$, we have \(v(x) = \sum_{y\in V}^{} \mathbb{P}_x(X_1=y) v(y) = \sum_{y\in V}^{} \frac{c(x,y)}{\sum_{z\in V} c(x,z) }v(y) = \sum_{y\in V}^{} c(x,y) \frac{v(y)}{c(x)}\) (harmonicity)
  \end{enumerate}
\end{lem}
\begin{proof}
    To show the harmonicity, we use the Markov property. Since \(x \neq a,z\), \(p^x(\tau_a\geq 1)=1\). The Markov property implies 
    \begin{align*}
      p^x(\tau_a \leq \tau_z) &= \mathbb{E}_x \left[ p^x(\tau_a < \tau_z)\right] \\
                              &= \sum_y p^x(X_1=y)\cdot p^y(\tau_a < \tau_z) \\
                              &= \sum_y p^x(X_1=y) \cdot v(y)
    \end{align*}    
\end{proof}

\begin{lem}[Uniqueness of voltage function]
    The voltage function $v$ is the only function satisfying the properties above.
\end{lem}
\begin{proof}
    Assume $v_1,v_2$ satisfy the properties of the voltage function. We show $h \equiv v_1-v_2=0$. 
    \begin{enumerate}
      \item By properties 1 and 2, $h(a)=h(z)=0$. 
      \item For $x\neq a,z$, we have \(h(x) = \sum_y \frac{c(x,y)}{c(x)} h(y)\) ($h(x)$ is a weighted average over its neighbors). Assume $h(x) \neq 0$. We assume \OE $h(x) >0$. Let \(M=\max_{}\set*{h(x); x\in V}>0\). Let \(H= \set*{x; h(x)=M}\). Note that $a,z \notin H$. Let $x$ be an element of $H$ with a neighbor $w$ outside of  $H$. Then, the following are true
        \begin{enumerate}
            \item $x\in H$, thus $h(x)=M$.
            \item 
              \begin{align*}
                h(x)  &=\sum_y \frac{c(x,y)}{c(x)} h(y) \\
                     &=\frac{c(x,w)}{c(x)}h(w) + \sum_{y\neq w} \frac{c(x,y)}{c(x)}h(y) \\
                     &\leq h(w) \frac{c(x,w)}{c(x)} + M \sum_{y\neq w} \frac{c(x,y)}{c(x)} \\
                     & h(w) \frac{c(x,w)}{c(x)} + M \left(1-\frac{c(x,w)}{c(x)}\right) \\
                     &=M + \underbrace{ \left(h(w)-M\right)}_{<0}\cdot \underbrace{\frac{c(x,w)}{c(x)}}_{>0} < M
              \end{align*}
              
        \end{enumerate}
        This is a contradiction to $h \neq 0$.
    \end{enumerate}
\end{proof}

\begin{remark}
    The proof is similar to the proof of uniqueness of the solution to the Dirichlet problem.
\end{remark}

\begin{remark}
    In the theory of electrical networks, the (physical) voltage function satisfies the same three properties (up to normalization of the boundary conditions). Therefore, everything that is known about physical electrical networks has the potential of being useful in the understanding of random walks on networks.
\end{remark}


\begin{defn}[Resistance]
    For an edge $e$, we define its resistance as
    \[r(e)= \frac{1}{e}\]
\end{defn}


\subsection{Currents and flows}

\begin{defn}[Current]
    Let $e=(x,y) \in E$. For every \(x,y\) we define the current as
    \[I(x,y) = \frac{v(x)-v(y)}{r(x,y} = \left(v(x)-v(y)\right) c(x,y)\]
\end{defn}


\begin{lem}[Properties of the current]
    Let $I$ be the current. Then, the following statements hold:
    \begin{enumerate}
      \item \(I(x,y) = -I(y,x)\) (Antisymmetry) 
      \item For $x\neq a,z$, $\sum_{y\neq x} I(x,y) = 0$ (Kirchhof's node law)
      \item For a cycle ${x}_{0} , \dots , {x}_{N}=x_0 $, then \(\sum_{n=1}^{N} I(x_{n-1},x_n)r(x_{n-1},x_n) = 0\) (Kirchhof's circuit law)
    \end{enumerate}
    
\end{lem}

\begin{proof} \,
    \begin{enumerate}
      \item follows from $c(x,y)=c(y,x)$.
      \item Notice that 
        \begin{align*}
          \sum_{y\neq x}I(x,y) &= \sum_{y\neq x}c(x,y) \left(v(x)-v(y)\right) \\
                               &=v(x)\sum_{y\neq x}c(x,y) - \sum_{y\neq x}c(x,y)v(y) \\
                               &=v(x)c(x) - c(x)\sum_{y\neq x}\frac{c(x,y)}{c(x)}v(y) \\
                               &=0
        \end{align*}
        where we used the harmonity of $v$.
      \item By definition of $I$, we have $(I\cdot r)(x,y)=v(x)-v(y)$. Thus, we have 
        \begin{align*}
          \sum_{n=1}^{N} I(x_{n-1,x_n)r(x_{n-1},x_n)} = \sum_{n=1}^{N} v(x_{n-1})-v(x_n) = 0
        \end{align*}        
    \end{enumerate}
\end{proof}


\begin{defn}[Flow]
    A flow is a function \(F: V^2\to \mathbb{R}\), such that 
    \begin{enumerate}
      \item \(F(x,y) = -F(y,x)\)
      \item If $x \nsim y$, then \(F(x,y)=0\)
      \item For all \(x\neq a,z\), we have \(\sum_{y\neq x} F(x,y)=0\) (Kirchhof's node law)
    \end{enumerate}    
\end{defn}


\begin{defn}[Total current]
    The total current in a network is defined as 
    \[\sum_{x\neq a} I(a,x)  =\sum_{x\neq z} I(x,z)\]
\end{defn}

\begin{proof}
    We prove equality for the two descriptions of total current.
    \begin{align*}
      \sum_{x\neq a} I(a,x) - \sum_{x \neq z} I(x,z) &= \sum_{x \neq a}^{} I(a,x) + \sum_{x \neq z}^{} I(z,x) + \underbrace{\sum_{y \neq a,z} \sum_{x \neq y} I(y,x)}_{=0} \\
                                                     & \stackrel{(\mathrm{I})}{=} \sum_{y} \sum_{x \neq y} I(y,x) \\
                                                     &=\sum_{\{x,y\} \ssq V} I(x,y) + I(y,x) \\
                                                     &= 0
    \end{align*}
\end{proof}


\begin{defn}[Effective resistance]
    The effective resistance in a network is defined as 
    \[R_{\mathrm{eff}}(a,z) = \frac{1}{\mathrm{total\;current}}\]
    Similarly, we denote
    \[c_{\mathrm{eff}}(a,z) = \mathrm{total\;current}\]
\end{defn}


\begin{nota}[Stopping times]
    We use the following notation for stopping times 
    \begin{align*}
      \tau_x  &= \inf \set*{t\geq0; X_t=x} \\
      \sigma_x &= \inf \set*{t\geq0; X_t\neq x} \\
      \eta_x^0 &=0 \\
      \eta_x &= \inf \set*{t>0; X_t=x} \\
      \eta_x^{n+1} &= \inf \set*{t > \eta_x^n; X_t=x }
    \end{align*}
    Notice that $\tau_x \neq \eta_x$ under $\PP_x$ and $\tau_x=\eta_x$ under $\PP_y$ for $y\neq x$.
\end{nota}

\begin{lem}[$n$-th return time of Markov chains]
    Notice that if $X$ is a Markov chain, then we have 
    \[\eta_x^{n+1} = (\tau_x \circ \theta_{\sigma_x} + \sigma_x) \circ \theta_{\eta_{x}^n} + \eta_{x}^n \]
\end{lem}


\begin{thm}[]

    Let \(G=(V,E,c)\) be a connected network, \(a\neq z \in V\). Let \(\tau_x:=\inf_{} \set*{n\geq 0 ; X_n=x}\) and \(\eta_{x}=\inf_{} \set*{n\geq 1 ; X_n=1}\). Then, 
    \[\mathbb{P}_a \left(\tau_z < \eta_a\right) = \frac{c_{\mathrm{eff}}(a,z)}{c(a)}\]
\end{thm}

\begin{proof}
    We have that 
    \begin{align*}
      \mathbb{P}_a(\tau_z < \eta_a) &= \sum_x \mathbb{P}_a(X_1=x)\mathbb{P}_x(\tau_z < \eta_a) \\
                                    &=\sum_x \frac{c(a,x)}{c(a)} \cdot (\underbrace{1}_{v(a)}-\underbrace{\mathbb{P}_x(\tau_a-\tau_z)}_{v(x)}) \\
                                    &=\frac{1}{c(a)} \sum_x c(a,x) \left(v(a)-v(x)\right) \\
                                    &=\frac{1}{c(a)}\sum_x I(a,x) \\
                                    &=\frac{c_{\mathrm{eff}}(a,z)}{c(a)}
    \end{align*}
    
\end{proof}


\begin{thm}[Commute time formula]
    We have that 
    \[\mathbb{E}_a \left[\tau_z\right] + \mathbb{E}_z \left[\tau_a\right] = 2 R_{\mathrm{eff}}(a,z) + \sum_{e\in E}c(e)\]
\end{thm}

\begin{proof} We first provide facts about Markov chains.
    \begin{enumerate}
      \item Every Markov chain on a finite state space has an invariant distribution. If the Markov chain is a random walk on a finite connected network, then 
        \[\mu(x)=\frac{c(x)}{2 \sum_{e\in E}c(e)}\]
        is the unique invariant distribution, since 
        \begin{enumerate}[i)]
          \item \(\mu \) is a distribution: \(\mu(x)\geq 0\) holds since $c\geq 0$, and 
            \begin{align*}\sum_x \mu(x) &= \frac{1}{2 \sum_{e\in E} c(e)}\sum_x \left(\sum_y c(x,y)\right) \\
              &= \frac{1}{2 \sum_{e\in E}c(e)} \cdot 2 \sum_{e\in E} c(e) \\
              &=1
            \end{align*}
          \item \(\mu\) is invariant: Let $X=_d \mu$, then for every $x$
            \begin{align*}
              \mathbb{P}_{\mu }&= \sum_y \mathbb{P}_{\mu }(X_0=y) \mathbb{P}_{\mu } \setround*{X_1 ;X_0=y} \\
                               &= \sum_y \mu(y) \cdot \frac{c(x,y)}{c(y)} \\
                               &=\frac{1}{2 \sum_e c(e)} \sum_y c(y) \frac{c(x,y)}{c(y)} \\
                               &=\frac{1}{2 \sum_e c(e)}c(x) \\
                               &= \mu(x)
            \end{align*}
        \end{enumerate}
      \item \(\mathbb{E}_x[\eta_x] = \frac{1}{\mu(x)}\). We sketch the proof of this claim: Let \(\mu \) be stationary and invariant distribution, i.e. \(X_n =_d \mu\) for all $n\geq 0$. Then, for large $N$, $\mu(x)\cdot N$ is approximately the number of times we are at $x$. Thus, for the average distance between appearances of $x$, we have  \(\EE[\eta_x^{n+1}-\eta_x^n]=\frac{1}{\mu(x)}\). This can be shown using the Markov property to derive $\eta_x^1=_d \eta_x^{n+1}- \eta_x^n$ and using the Law of Large Numbers. 
      \item Let \(\eta_a^1, \eta_a^2, \dots\) be the consecutive return times to \(a\). Let \(m:=\inf_{} \set*{k\geq 0; \eta_a^{k-1}< \tau_z < \eta_a^k}\). Then
        \begin{align*}
          \mathbb{E}_{a}[\tau_z] + \mathbb{E}_{z}[\tau_a] &= \mathbb{E}_{a}[\eta_a^m] \\
                                                          &=\mathbb{E}_{a}[\eta_a^{m-1}]+ \mathbb{E}_a[\eta_a^m - \eta_a^{m-1}] \\
                                                          &=\mathbb{E}_a[m-1]\mathbb{E} \setsquared*{\eta_a^1 ; \eta_{a}^1 < \tau_z} + \mathbb{E}_{a} \setsquared*{\eta_a^1 ;\tau_z < \eta_a^n} \\ &= \mathbb{E}[\mathrm{Geom}(\mathbb{P}_{a}(\tau_z<\eta_A)-1] \cdot \mathbb{E}_{a} \setsquared*{\eta_a^1 ;\eta_a^1 < \tau_z} + \frac{\mathbb{P}_{a}(\tau_z<\eta_a^1)}{\mathbb{P}_a(\tau_z< \eta_a^1)} \mathbb{E}_a \setsquared*{\eta_a^1 ;\tau_z < \eta_a^1} \\
                                                          &=\left(\frac{1}{\mathbb{P}_a(\tau_z < \eta_a^1)}-1\right) \mathbb{E}_a \setsquared*{\eta_a^1 ;\eta_a^1< \tau_z} + \frac{\mathbb{P}_{a}(\tau_z<\eta_a^1)}{\mathbb{P}_a(\tau_z< \eta_a^1)} \mathbb{E} \setsquared*{\eta_a^1  ;\tau_z < \eta_a^1} \\
                                                          &=\frac{1}{\mathbb{P}_a (\tau_z< \eta_a^1)} \left[\mathbb{P}_a(\eta_a^1<\tau_z) \mathbb{E}_a \setsquared*{\eta_a^1 ;\eta_a^1 < \tau_z}+ \mathbb{P}_a(\tau_z<\eta_a^1)  \mathbb{E}_a \setsquared*{\eta_a^1 ;\tau_z < \eta_a}\right] \\
                                                          &=\frac{1}{\mathbb{P}_a(\tau_z < \eta_a^1} \mathbb{E}_a[\eta_a^1] \\
                                                          &= \frac{2 \sum_e c(e)}{c(a)} / \frac{c_{\mathrm{eff}(a,z)}}{c(a)} \\
                                                          &=2 \frac{\sum_e c(e)}{c_{\mathrm{eff}}(a,z)} \\
                                                          &=2 R_{\mathrm{eff}(a,z)} \sum_e c(e)
        \end{align*}
        
    \end{enumerate}

\end{proof}



\begin{exer}
    We have that 
  \[\mathbb{P}_x(\tau_a< \tau_z) \leq \frac{c_{\mathrm{eff}}(x,a)}{c_{\mathrm{eff}}(x,z)}\]
\end{exer}

To make use of these theorem, we need to be able to estimate (or calculate) effective resistance/conductance. In some examples we are capable of calculating effective resistances.

\begin{fact}
    We have the following: 
    \begin{enumerate}[1)]
      \item in $d=1$, \(R_{\mathrm{eff}}(0,n)=n\)
    \item in $d=2$, \(R_{\mathrm{eff}} \left((0,0),(n,n)\right)=\mathcal{O}(\ln n)\)
    \item in $d=3$, \(R_{\mathrm{eff}} \left((0,0,0),(n,n,n)\right)=\mathcal{O}(n) \)
    \end{enumerate}
Thus, we have 
\begin{align*}
  \mathbb{E}_{a,0}[\tau_{n,n}] &= \frac{1}{2} \left(\mathbb{E}_{a,0}[\tau_{n,n}]+ \mathbb{E}_{n,n}[\tau_{0,0}]\right) \\
                               &=\frac{1}{2} 2 R_{\mathrm{eff}} \left((0,0),(n,n)\right) \sum_e c(e) \\
                               &=\mathcal{O}(n^2 \cdot \ln n)
\end{align*} 

\end{fact}


\begin{defn}[Associated Hilbert space]
    Let \(G=(V,E)\) be a graph of bounded degree, i.e. \(\sup_{v\in V} \operatorname{deg}v<\infty \). We define the associated Hilbert space 
    \[\ell^2(V) := \set*{f: V \to \mathbb{R}; \sum_v f^2(v) < \infty }\]
with the scalar product 
\[\left\langle f, g \right\rangle =  \sum_v f(v)\cdot g(v) \quad \text{for \(f,g \in \ell^2(V).\)}\]
For \(e=(u,v)\), we define \(\widecheck{e}=u, \widehat{e}=v\) and \(-e=(v,u)\). We define the associated Hilbert space 
\[\ell^2_{-}(E):= \set*{\theta: E \to \mathbb{R}; \sum_e \theta(e)< \infty,\, \theta(-e)=-\theta(e) }\]
with the scalar product
\[ \left\langle \theta, \eta \right\rangle = \frac{1}{2} \sum_e \theta(e) \cdot \eta(e)\]

\end{defn}

\begin{defn}[Boundary and coboundary operator]
We define the \emph{coboundary operator} as 
\[d: \ell^2(V) \to \ell^2_{-}(E),\, df(e)=f(\widecheck{e})-f(\widehat{e}) \quad\text{for \(e\in E\).}\]
and the \emph{boundary operator} as 
\[d^*: \ell^2_{-}(E) \to \ell^2(V),\, d^*\theta(v)=\sum_{e: \widecheck{e}=v} \theta(e) \quad\text{for \(v\in V\).} \]
   

\end{defn}

\begin{exer}[]
    Show that $d,d^*$ are well-defined.
\end{exer}

\begin{lem}[]
    \(d,d^*\) are adjoint linear maps, i.e. for all \(f \in \ell^2(V), \theta \in \ell^2_{-}(E)\), we have \( \left\langle \theta, df \right\rangle = \left\langle d^*\theta, f \right\rangle  \).
\end{lem}

\begin{proof}
    We show the proof in two steps:
    \begin{enumerate}[]
      \item \(- \sum_{e: \widehat{e}=v}\theta(e) = \sum_{e: \widehat{e}=v} \theta(-e) = \sum_{e: \widecheck{e}=v} \theta(e) = d^*\theta(v)\).
      \item By definition of the scalar product, we have
        \begin{align*}
          \left\langle \theta,df \right\rangle &= \frac{1}{2}\sum_e \theta(e) \cdot \left(f( \widecheck{e})- f( \widehat{e}) \right) \\
                                               &=\frac{1}{2}\sum_e \theta(e) f( \widecheck{e}) - \frac{1}{2} \sum_e \theta(e) f( \widehat{e}) \\
                                               &=\frac{1}{2}\sum_v f(v) \sum_{e: \widecheck{e}=v}\theta(e) - \frac{1}{2} \sum_v f(v)\sum_{e: \widehat{e}=v} \theta(v) \\
                                               &\overset{\mathclap{1.}}{=} \sum_v f(v)\cdot d^*\theta(v) \\
                                               &= \left\langle d^*\theta, f \right\rangle 
        \end{align*}
    \end{enumerate}
    
\end{proof}


\begin{remark}[]
    We have the following results:
    \begin{itemize}
      \item Ohm's law:  \(dU(e) = i(e)\cdot r_{e}\);
      \item Kirchhoff's node law: \(d^*i(v) = 0\) for \(v \in V \setminus (A \cup Z)\).
    \end{itemize}
    
\end{remark}

\begin{defn}[Flow and strength]
   \(\theta\) is called a \emph{flow} ffrom \(A\) to \(Z\) if 
   \begin{enumerate}[1)]
     \item \(d^*\theta \geq 0\) on \(A\);
     \item \(d^*\theta =0\) on \(V \setminus (A\cup Z)\); 
     \item \(d^*\theta\leq 0\) on \(Z\).
   \end{enumerate}
   We define the \emph{strength} of a flow \(\theta\) as \( \operatorname{strength}(\theta) = \sum_{a\in A} d^*\theta(a)  \).
\end{defn}

\begin{lem}[Flow conversion]
    Let \(G\) be a finite graph and \(theta\) be a flow from \(A\) to \(Z\) with \(A,Z \subseteq  V\) and \(A \cap Z = \emptyset \). Then, 
    \[\sum_{a \in A} d^* \theta(a) = - \sum_{z\in Z} d^*\theta(z)\]
    Furthermore, if \(f:V \to \mathbb{R}\) such that \(f\vert_A = \alpha \) and \(f\vert_Z=\zeta\), then \( \left\langle \theta, df \right\rangle = \operatorname{strength}(\theta)(\alpha-\zeta)\).
\end{lem}


\begin{proof}
   We show the proof in two steps:
  \begin{enumerate}[]
    \item We have by defintion of the scalar product
      \begin{align*}
        \sum_{a\in A} d^* \theta(a) + \sum_{v\in V \setminus (A\cap Z)} d^*\theta(v) + \sum_{z\in Z} d^*\theta(z) &= \left\langle d^*\theta, \mathbbm{1}_V \right\rangle \\
                                                                                                                  &= \left\langle \theta, d \mathbbm{1}_V \right\rangle \\
                                                                                                                  &=0
      \end{align*}
    \item To show the second statement, notice that 
      \begin{align*}
        \left\langle \theta, df \right\rangle &= \left\langle d^*\theta, f \right\rangle \\
                                              &=\sum_{v\in V} d^*\theta(v)\cdot f(v) \\ 
                                              &=\sum_{a\in A} d^*\theta(a)\cdot f(a) + \sum_{z\in Z} d^*\theta(z)\cdot f(z) \\
                                              &\overset{\mathclap{1.}}{=} \operatorname{strength}(\theta)(\alpha-\zeta)
      \end{align*}
      
  \end{enumerate}
  
\end{proof}

\begin{defn}[Energy]
    For antisymmetric fucntions \(\theta, \eta\), we define the scalar product
    \[ \left\langle \theta, \eta \right\rangle_r = \frac{1}{2}\sum_{e}\theta(e)\cdot \eta(e)\cdot r_{e}\]
    We define the \emph{energy} of an antisymmetric function \(\theta\) as \(\mathcal{E}(\theta)=\norm{\theta}_{r}^2 = \frac{1}{2}\sum_{e}\theta(e)\cdot r_{e}\).
\end{defn}


\begin{lem}[]
    Let \(i: E \to \mathbb{R}\) be a unit current flow from \(A\) to \(Z\) such that the voltage \(\mathcal{U}\) satisfies \(\mathcal{U}\vert_A= \mathcal{U}_A\) and \(\mathcal{U}\vert_Z=\mathcal{U}_Z\). Then, \(\mathcal{E}(i)=\mathcal{U}_A- \mathcal{U}_Z= R_{\mathrm{eff}}(A \leftrightarrow Z)\). 
\end{lem}

\begin{proof}
    By Ohm's law, we have 
    \begin{align*}
      \mathcal{E}(i) &= \left\langle i,i \right\rangle_r \\
                     &= \left\langle i,ri \right\rangle \\
                     &= \left\langle i,dU \right\rangle \\
                     &= \left(\mathcal{U}_A- \mathcal{U}_Z\right) \underbrace{\operatorname{strength}(i)}_{=1} \\
                     &=\mathcal{U}_A-\mathcal{U}_Z
    \end{align*}
    Since all vertices in \(A\) have the same voltage, we can identify them as one point \(a\). Then,
    \begin{align*}
      c_{\mathrm{eff}}(a \leftrightarrow Z) &= \sum_{x\neq a} \theta(a,x) \\
                                            &= \sum_{x\neq a} \frac{i(a,x)}{\mathcal{U}_A-\mathcal{U}_Z} \\
                                            &= \frac{1}{\mathcal{U}_A-\mathcal{U}_Z}
    \end{align*}
    This is equivalent to \(R_{\mathrm{eff}}=\mathcal{U}_A \cdot \mathcal{U}_Z\).    
\end{proof}

\begin{lem}[]
    Let \(i: E \to \mathbb{R}\) be antisymmetric satisfying Kirchhoff's cycle law. Suppose \(i\) satisfies Kirchhoff's node law on \(W \subseteq V\). Then, there exists a function (voltage) \(\mathcal{U}: V \to \mathbb{R}\), such that \(U\) is harmonic on \(W\) and \(i\) is the current associated with \(\mathcal{U}\). The voltage is unique up to addition. 
\end{lem}

\begin{proof}
    We show the proof in two steps:
    \begin{enumerate}[]
      \item Existence: Let \(v_0\in V\) and set \(\mathcal{U}(v_0)=0\). For \(u\in V\), let \((v_0,\dots, v_N=u)\) be a path from \(v_0\) to \(u\). We define \[\mathcal{U}(u)=\sum_{n=1}^{N}r_{\{v_{n-1},v_{n}\}}i(v_{n}, v_{n-1}) \]
        Let \((w_0=v_0, \dots, w_m=v_n)\) be a second path, then \((v_0, \dots, v_n=w_m, \dots, w_0=v_0)\) is a cycle, Thus, 
        \[\sum_{n=1}^{N} r_{\{v_{n-1},v_n\}} i(v_{n},v_{n-1})=0 \]
        Thus, \(\mathcal{U}\) is well-defined. For \(\{u,v\} \in E\), we can join the edge to the path \[\mathcal{U}(v) = \mathcal{U}+r_{\{u,v\}}i(v,u)\]
        Then, we have \(i(v,u) = \frac{1}{r_{\{u,v\}}} \left(\mathcal{U}(v)-\mathcal{U}(u)\right)\). Then, the current \(i\) is associated to \(\mathcal{U}\). Harmonicity on \(W\) follows from Kirchhoff's node law.
      \item Uniqueness: For a given \(\mathcal{U}(v_0)\), we can define \(\mathcal{U}(u)\) inductively by Ohm's law. 
    \end{enumerate}
    
\end{proof}

