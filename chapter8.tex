\section{Poincare inequality}

\subsection{Classic example}
A classic problem is given by the following settung. Let $\Omega\ssq \RR^n$ be an open, bounded, connected and convex, $f:\overline{\Omega}\to \RR$ be smooth, i.e. $f\in W^{1,2}$ with $\int_\Omega f \dd{x}=0$
We are interested to find the largest constant $\kappa(\Omega)$ that satisfies

$$\kappa(\Omega) \int_\Omega f^2 \dd{x} \leq \int_\Omega \abs{\nabla f}^2 \dd{x}$$

We integrate using the usual Lebesgue measure and not $\dd{x}$ instead of $\dd{\lambda(x)}$. We also set $\operatorname{diam(\Omega)} = \max_{(x,y)\in \partial \Omega} d(x,y)$. The problem is the solved by the Poincare constant $\kappa(\Omega)=\frac{\pi^2}{\operatorname{diam}(\Omega)}^2$ !!! Reference (Payne, Weinberger 60?) !!!

The problem is equivalent to 
$$\kappa(\Omega) \VV_\Omega[f] \leq \int_\Omega \abs{\nabla f}^2 \frac{\dd{x}}{\abs{\Omega}}$$
with $\VV_\Omega(f)=\int_\Omega f^2 \frac{\dd{x}}{\abs{\Omega}} - \left(\int_\Omega f \frac{\dd{x}}{\abs{\Omega}}\right)^2$


\subsubsection{Relation to spectral theory of the Laplacian}


We consider the quadratic form $\EEE(f)= \int_\Omega \abs{\nabla f}^2 \dd{x}$ for $f\in W^{1,2}(\Omega)$. $\EEE$ is non-negative and closed. By Friedrich's extension theorem, it defines a non-negative self-adjoint operator $-L=-\Delta$, i.e. there is a unique $-\Delta$ with 
$$\EEE(f) = \int_\Omega f(-\Delta f) \dd{x} \quad (n\cdot \nabla f\vert_{\partial \Omega}=0)$$
The eigenvalues of $-\Delta$ form an countably infinite, non-negative sequence $(\lambda_n)_n$ starting at zero, i.e $0=\lambda_0<\lambda_1<\dots$.
The "first" non-zero eigenvalue $\lambda_1$ is called spectral gap of $L$ ($\Delta$). An eigenfunction for $\lambda_0=0$ is given by $\psi_0\equiv 1$. It can be shown that a min-max principle for the eigenpairs holds, i.e
$$\lambda_1 = \inf_{\substack{f\in W^{1,2}(\Omega) \\ \int_\Omega f \dd{x}=0}} \frac{\EEE(f,f)}{\int_\Omega f^2 \dd{x}}$$
and more generally !!! Principle !!!. By this principle, we know that $\kappa(\Omega)=\lambda_1$. 


\begin{exm}[Significance for semigroup generated by $L$]
    Let $P_t = e^{tL}$ be the semigroup generated by $L$.
    Then, regarding Ergodicity of $P_t$, it holds: \quad $P_tf=f \iff f = \operatorname{const.} $ We can quantify the rate of convergence to the equilibrium using an estimation from above: \quad $\norm{P_tf}_2 \leq e^{-t\lambda_1}\norm{f}_2 \quad (\int_\Omega f \dd{x}=0)$. This holds due to $P_t f = \sum_{n=1}^\infty e^{t\lambda_n} \psi_n \sqb{\psi_n,f}$.
    
\end{exm}


\subsection{General Markov semigroups}

\begin{defn}[Semigroup]

\end{defn}


Let $(\Omega,\AAA, \mu)$ be a probability space and $P_t = e^{tL}$ $(t\geq0)$ a strongly continuous semigroup of contrations on $\LLL^2(\mu)$ and $L:\DDD(L)\to \LLL^2(\mu)$ be its generator.


\begin{defn}[Reversibility]

\end{defn}


\begin{thm}[Reversibility]
The semigroup is reversible if $\sqb{f,P_t g} = \sqb{P_tf,g}$ iff $L$ is self-adjoint
\end{thm}


The Dirchlet form corresponding to $L$ is given by $$\EEE(f,g):= -\sqb{f,Lg}$$
for $f\in \LLL^2(\mu),g\in \DDD(L)$.


\begin{thm}[Equivalences for ergodic and reversible $(P_t)_t$]
    Let $(P_t)_t$ be ergodic and reversible. Then the following statements are equivalent:
    \begin{enumerate}
        \item The Poincare inequality holds, i.e. $\VV_\mu[f]:=\int_\Omega f^2 \dd{\mu}- \left(\int_\Omega f \dd{\mu}\right)^2 \leq c \EEE(f,f)$ for all $f$
        \item $\norm{P_t f - \mu(f)}\leq e^{-\frac{t}{c}}\norm{f-\mu(f)}_2$ for all $t\geq 0$ and $f$
        \item $\EEE(P_tf,P_tf) \leq e^{-2\frac{t}{c}E(f,f)}$ for all $t\geq 0$ and $f$
    \end{enumerate}
\end{thm}


\begin{exm}[Markov chain on finite state space]
    Let $\Omega = \cub{1,\dots, d}$, $\mu$ be a stationary distribution on $\Omega$. Let $K:\Omega \times \Omega \to [0,\infty)$ be a transition kernel on $\Omega$, i.e. $\sum_{y\in \Omega}K(x,y)=1$. Assume $K$ is reversible wrt to $\mu$, i.e. $\mu(x)K(x,y) = \mu(y)K(y,x)$ for all $x,y \in \Omega$. Consider the Dirichlet form with generator $L=K-\ONE$. Then
    \begin{align*}
        E(f,f) &= -\int_\Omega Lf \dd{\mu} = - \sum_{x} f(x) \left[\sum_{y}K(x,y) f(y)-f(x)\right] \pi(x) \\
        &=\frac{1}{2} \sum_{x,y} \left[f(x)-f(y)\right]^2 K(x,y)\pi(x)
    \end{align*}
    Thus, we have 
    \begin{align*}
        \VV_\mu[f] &=\sum_x f(x)^2 \pi(x) - \left(\sum_x f(x) \pi(x)\right)^2 \\
        &= \frac{1}{2} \sum_{x,y} (f(x)-f(y))^2 \pi(x)\pi(y)
    \end{align*}
\end{exm}


\begin{exm}[Gaussian Poincare inequality in one dimension]
    Let $\dd{\mu(x)} = e^{-\frac{x^2}{2}\frac{\dd{x}}{\sqrt{2\pi}}}$. We define 
\end{exm}


\begin{thm}[Tensorization of variance]
    For independent random variables $\XN$ and any $f: \Omega_1 \times \dots \times \Omega_N \to \RR$, we have 
    $$\VV f(\XN) \leq \EE[\sumn  \VV_n[f(\XN)]]$$
    where $\VV_n[f(\xN)]:= \VV[f(x_1,\dots,x_{n-1},X_n,x_{n+1}, \dots ,x_n)]$
\end{thm}


\begin{thm}[Tensorization of Poincare inequality]
    Let $(\Omega_n,\AAA_n, \mu_n)$ satisfy Poincare inequalities with the same constant $\kappa>0$, i.e. for all $f\in \LLL^2(\mu_j)$, 
    $$\kappa \VV_n(f) \leq \int_{\Omega_n} \abs{\nabla f}^2 \dd{\mu_n}$$
    then $\bigotimes_{n=1}^N (\Omega_n, \AAA_n, \mu_n)$ satisfies the Poincare inequality with the same constant, i.e. for all $f\in \LLL^2(\mu)$, where $\mu=\bigotimes_{n=1}^N \mu_n$
    $$\kappa \VV_{\mu}(f) \leq \int_\Omega \abs{\nabla f}^2 \dd{\mu}$$
\end{thm}


\begin{exm}[Poincare inequality for independent standard normal random variables]
    Let $\XN$ be independent with distribution $\NNN(0,1)$, and $f\in \LLL^2$. Then, 
    $$\VV f(\XN) \leq \EE\abs{\nabla(\XN)}^2$$
\end{exm}